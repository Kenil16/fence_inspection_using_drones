\documentclass[../Head/Main.tex]{subfiles}
\begin{document}
\section{Conclusion}
\label{sec:conclusion}

A simulation of the drone and optitrack system were setup and used to develop the initial automated controller for the drones trajectory. This system is fully setup with ROS nodes to be able to utilise the offboard controller node, however no complete test on the actual system was performed due to time constraints. 

Waypoints were selected manually besides the fence and connected for the drone to follow. From these points a mission was made utilising QGroundControl as the GUI. 

We were able to start the mission as well as the video capture node running on the Pi, directly from the pilots controller, to the receiving unit on the drone. This setup worked quite well for doing multiple tests of the system. 

\par 
Multiple different vision approaches were analysed and the mask-rcnn algorithm was found to be the best performing one to segment and find a breach in a fence. The network was trained with an augmented dataset of 5580 images which consisted of artificial snow, rain, fog and other kinds of noise in the images to replicate real-life scenarios. The network trained with this augmented dataset yielded quite good results with an mAP of 0.874 in the augmented test set of 1395 images.

This algorithm was able to find all custom made breaches (cardboard) placed on the fence in the final acceptance test. However, also false positives were found as breaches. This suggests that improvements have to be made on training the network with a bigger dataset for future use as well as changes to the camera setup and distance away from the fence.
\\
The SORA process helped us in identifying the challenges in operation and design of the drone. This SORA will be presented to the authorities for the approval of the fence inspection drone operation.

\end{document}